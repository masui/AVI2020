% AVI2020

\documentclass[sigconf]{acmart}

\settopmatter{printacmref=false} % アブストラクトの下のRef情報を表示しない

\usepackage{graphicx}
\usepackage{here} % [H]とするとその場所に配置されるらしい

% Copyright
%\setcopyright{none}
%\setcopyright{acmcopyright}
%\setcopyright{acmlicensed}
\setcopyright{rightsretained}
%\setcopyright{usgov}
%\setcopyright{usgovmixed}
%\setcopyright{cagov}
%\setcopyright{cagovmixed}


% % DOI
% \acmDOI{10.475/123_4}
% 
% % ISBN
% \acmISBN{123-4567-24-567/08/06}
% 
% Conference
\acmConference[AVI2020]{International Conference on Advanced Visual Interfaces}{May 2020}
              {Island of Ischia (Italy)}
\acmYear{2020}
\copyrightyear{2020}
\acmISBN{}
\acmDOI{}

\begin{document}
\title{DrawWiki: a drawing-based Wiki system}

% \titlenote{Produces the permission block, and copyright information}

% \author{Takumi Hayakawa}
\author{Takumi Hayakawa,\hspace{1em} Toshiyuki Masui}
% \affiliation{%
%   \vspace{-1em}\institution{Aliabad State University$^*$ \hspace{0.3em} North Laboratory$^\mathsection$}
% }
% \authornote{Is this necessary?.}
% \orcid{1234-5678-9012}
\affiliation{%
  \institution{Keio University}
  \streetaddress{5322 Endo}
  \city{Fujisawa}
  \state{Kanagawa}
  \postcode{252-8520}
}
\email{takumin@takumin.om, masui@pitecan.com}

\renewcommand{\shortauthors}{T. Hayakawa}

\begin{CCSXML}
  <ccs2012>
  <concept>
  <concept_id>10002978.10002991.10002992.10011618</concept_id>
  <concept_desc>Security and privacy~Graphical / visual passwords</concept_desc>
  <concept_significance>500</concept_significance>
  </concept>
  </ccs2012>
\end{CCSXML}
\ccsdesc[500]{Security and privacy~Graphical / visual passwords}

\keywords{Passwords; visual passwords; user authentication;
  episodic memories; draw-a-secret, EpisoPass, EpisoDAS.}

\begin{abstract}

We propose a drawing-based note-taking style where users can use
handwritten objects not only for showing shapes and texts, but for
linking objects just like hyperlink texts are used for linking pages.

Hypertexts are widely used on wiki systems like Wikipedia, where words
and phrases are used for linking pages. Although wiki systems are
useful for managing a large amount of text data, it is not possible to
use non-text data for linking information. It would be more useful if
handwritten drawings can also be used as hyperlinks on wiki pages just
like textual phrases are used for linking pages. To prove the concept
of drawing-based wiki systems, we have implemented “DrawWiki,
where arbitrary handwritten drawings can be used as
links to other pages and objects.

In this paper, we describe the design, implementation, evaluations and
applications of DrawWiki, and discuss the future of wiki systems where
the mixture of texts and drawings are used as hyperlinks.

\end{abstract}

\maketitle

\section{Introduction}

\section{DrawWiki}

\section{Conclusion}

\large
\bibliographystyle{ACM-Reference-Format}
\bibliography{demo}

\end{document}
